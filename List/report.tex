\documentclass[UTF8]{ctexart}
\usepackage{geometry, CJKutf8}
\geometry{margin=1.5cm, vmargin={0pt,1cm}}
\setlength{\topmargin}{-1cm}
\setlength{\paperheight}{29.7cm}
\setlength{\textheight}{25.3cm}

% useful packages.
\usepackage{amsfonts}
\usepackage{amsmath}
\usepackage{amssymb}
\usepackage{amsthm}
\usepackage{enumerate}
\usepackage{graphicx}
\usepackage{multicol}
\usepackage{fancyhdr}
\usepackage{layout}
\usepackage{listings}
\usepackage{float, caption}

\lstset{
    basicstyle=\ttfamily, basewidth=0.5em
}

% some common command
\newcommand{\dif}{\mathrm{d}}
\newcommand{\avg}[1]{\left\langle #1 \right\rangle}
\newcommand{\difFrac}[2]{\frac{\dif #1}{\dif #2}}
\newcommand{\pdfFrac}[2]{\frac{\partial #1}{\partial #2}}
\newcommand{\OFL}{\mathrm{OFL}}
\newcommand{\UFL}{\mathrm{UFL}}
\newcommand{\fl}{\mathrm{fl}}
\newcommand{\op}{\odot}
\newcommand{\Eabs}{E_{\mathrm{abs}}}
\newcommand{\Erel}{E_{\mathrm{rel}}}

\begin{document}

\pagestyle{fancy}
\fancyhead{}
\lhead{楠迪, 3230101283}
\chead{数据结构与算法第四次作业}
\rhead{Oct.21th, 2024}

\section{测试程序的设计思路}

\LARGE{1.初始化功能测试} \\
\large{}
我首先创建了一个链表,测试了默认构造函数和push与size函数,创建一个空的对象 myList。输出是否成功初始化为空列表。
使用 pushback 方法依次添加元素 10, 20, 30 到列表尾部。输出列表当前大小。
使用 front和back方法获取并输出列表的第一个和最后一个元素。
使用 pushfront 方法在列表头部插入元素 0。输出当前的第一个元素。\\
\LARGE{2.迭代器功能测试} \\ 
\large{}
创建一个迭代器 it 指向列表的开始。
输出当前指向的元素,并进行前置递增。
输出前置递增后指向的元素。\\ 
利用迭代器进行遍历。\\ 
\LARGE{3.list函数功能测试} \large{}\\ 
使用 popfront 删除头部元素,并输出删除操作信息。
使用 popback 删除尾部元素,并输出删除后的列表大小。\\ 
再次遍历\\ 
让迭代器指向第二个元素位置,并使用 insert 方法在此位置插入元素 15。输出插入后的列表内容。
随后在边界位置插入新元素5 和 20\\ 
让迭代器指向第二个元素,并使用 erase 方法删除该元素。输出删除后的列表内容。\\ 
测试拷贝构造函数。创建 copyList,使用 myList 初始化。遍历并输出拷贝后的列表内容。\\ 
初始化列表构造函数测试,创建一个新的列表 initList,使用初始化列表 {1, 2, 3, 4, 5}。
输出初始化列表的内容。\\ 
测试调用 clear 方法清空 myList。输出清空后的列表状态(是否为空)。



\section{测试的结果}

测试结果一切正常。\\ 
\small{}初始化一个空的 List: 成功\\
向 List 尾部插入元素 10, 20, 30\\
List 的大小: 3\\
List 的第一个元素: 10\\
List 的最后一个元素: 30\\
向 List 头部插入元素 0\\
List 的第一个元素: 0\\
前置递增前指向的元素: 0\\
前置递增后指向的元素: 10\\
后置递增前指向的元素: 10\\
后置递增后指向的元素: 20\\
前置递减后指向的元素: 10\\
后置递减前指向的元素: 10\\
后置递减后指向的元素: 0\\
当前迭代器解引用的元素: 0\\
it 和 it2 是否相等 (预期: 是): 是\\
it 和 it2 是否相等 (预期: 否): 否\\
正向遍历 List: 0 10 20 30 \\
删除 List 的头部元素\\
删除 List 的尾部元素\\
删除后的 List 大小: 2\\
删除后遍历 List: 10 20 \\
在第二个元素前插入 15 后的 List: 10 15 20 \\
删除第二个元素后的 List: 10 20 \\
拷贝后的 List: 10 20 \\
使用初始化列表构造的 List: 1 2 3 4 5 \\
head 指针的地址: 1\\
清空 List 后,List 是否为空: 是\\

\Large{我用 valgrind 进行测试,发现没有发生内存泄露。}
\large{}
\section{(可选)bug报告}

我发现了一个 bug,触发条件如下:

\begin{enumerate}
    \item 首先……
    \item 然后……
    \item 此时发现……
\end{enumerate}

据我分析,它出现的原因是:

\end{document}

%%% Local Variables: 
%%% mode: latex
%%% TeX-master: t
%%% End: 